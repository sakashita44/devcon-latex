\documentclass[a4paper,12pt]{ujarticle}

% upLaTeX用の日本語パッケージ
\usepackage[utf8]{inputenc}
\usepackage{otf}
\usepackage{amsmath,amssymb,amsfonts}
\usepackage{graphicx}
\usepackage[dvips]{color}
\usepackage{url}
\usepackage{geometry}
\geometry{margin=2.5cm}

% 画像パス設定
\graphicspath{{./figures/}{../figures/}}

% タイトル設定
\title{upLaTeX用サンプル文書}
\author{テスト作成者}
\date{\today}

\begin{document}

\maketitle

\section{はじめに}

これはupLaTeXの自動ビルド機能をテストするために作成されたサンプル文書です。

\section{数式}

簡単な数式の例:
\begin{equation}
    E = mc^2
\end{equation}

インライン数式の例: $\int_{-\infty}^{\infty} e^{-x^2} dx = \sqrt{\pi}$

\section{リスト}

\subsection{箇条書き}
\begin{itemize}
    \item 第一項目
    \item 第二項目
    \item 第三項目
\end{itemize}

\subsection{番号付きリスト}
\begin{enumerate}
    \item 最初の手順
    \item 二番目の手順
    \item 三番目の手順
\end{enumerate}

\section{表}

\begin{table}[h]
    \centering
    \begin{tabular}{|c|c|c|}
        \hline
        列1 & 列2 & 列3 \\
        \hline
        A  & B  & C  \\
        D  & E  & F  \\
        \hline
    \end{tabular}
    \caption{サンプル表}
    \label{tab:sample}
\end{table}

\section{日本語の特徴}

upLaTeXでは以下のような日本語の特徴を適切に処理できます:

\begin{itemize}
    \item ひらがな、カタカナ、漢字の混在
    \item 句読点の適切な配置
    \item 日本語と英語の混在: Hello World こんにちは世界
    \item 数式との組み合わせ: 円周率は$\pi \approx 3.14159$である
\end{itemize}

% 章ファイル読み込み
\section{Example Chapter}

\subsection{Introduction}

This chapter demonstrates the use of images and bibliography in pdfLaTeX documents with best practices.

\subsection{Image References}

Figure~\ref{fig:test_image_pdf} shows a test image. This image is loaded using relative path reference through \texttt{graphicspath} configuration.

\begin{figure}[htbp]
    \centering
    \includegraphics[width=0.6\textwidth]{figures/test/test.png}
    \caption{Example test image}
    \label{fig:test_image_pdf}
\end{figure}

The image path is resolved automatically using the \texttt{\textbackslash graphicspath} setting in the preamble, following best practices.

\subsection{Bibliography Citations}

This section demonstrates bibliography citation examples. For instance, we reference the work by Doe~\cite{example}.

Bibliography management is handled automatically through .latexmkrc configuration.

\subsection{Summary}

This chapter implements LaTeX document creation best practices including:

\begin{itemize}
    \item Image path management with \texttt{graphicspath}
    \item Portability through relative path references
    \item BibTeX bibliography management
\end{itemize}


\section{参考文献}

upLaTeXの詳細については、日本語LaTeXの公式ドキュメントを参照してください。

% 参考文献
\bibliographystyle{plain}
\bibliography{reference}

\section{まとめ}

この文書はupLaTeXで正常にコンパイルされ、LaTeX Workshopの自動ビルド機能のテストに使用できます。

\end{document}
