\documentclass[12pt]{ltjsarticle} % ドキュメントクラスの指定

% その他のパッケージ
\usepackage[margin=2cm]{geometry} % 余白の設定
\usepackage{amsmath} % 数式のサポート
\usepackage{graphicx} % 画像の挿入

\usepackage{RSL_style} % 独自のスタイルファイルを読み込み

\title{明朝体とゴシック体のサンプル文書}
\author{研究者}
\date{\today}

\begin{document}

\maketitle

\section{はじめに}

この文書は明朝体とゴシック体を使い分けたLaTeX文書のサンプルです。

\section{フォントの特徴}

明朝体の特徴:
\begin{itemize}
    \item 本文に適している
    \item 長文でも読みやすい
    \item 伝統的で落ち着いた印象
\end{itemize}

{\sffamily
この部分はゴシック体で表示されます。ゴシック体の特徴:
\begin{itemize}
    \item 見出しに適している
    \item 強調したい部分に使用
    \item 視認性が高い
\end{itemize}
}

通常の明朝体に戻ります。

\section{数式の例}

数式もきれいに表示されます:

\begin{equation}
    E = mc^2
\end{equation}

\begin{align}
    \sin^2\theta + \cos^2\theta &= 1 \\
    \frac{d}{dx}\sin x &= \cos x
\end{align}

\section{表の例}

\begin{table}[h]
\centering
\begin{tabular}{|c|c|c|}
\hline
項目 & 値 & 備考 \\
\hline
温度 & 25°C & 室温 \\
湿度 & 60\% & 標準 \\
圧力 & \SI{101.3}{kPa} & 標準大気圧 \\
\hline
\end{tabular}
\caption{実験条件}
\end{table}

\section{フォント使い分けの例}

通常の明朝体: あいうえおABCabc

{\sffamily ゴシック体: あいうえおABCabc} % 日本語部分はゴシック体にならない

{\sffamily\bfseries ゴシック体太字: あいうえおABCabc} % 日本語部分もゴシック体太字になる

\section{まとめ}

明朝体を基本として、必要に応じてゴシック体を使い分けることで、読みやすい文書を作成できます。

\end{document}
